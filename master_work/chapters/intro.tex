\indent
\indent
С каждым днем пользователи социальных сетей создают и потребляют все б\'{о}льшие
и б\'{о}льшие объемы информации, в том числе огромное количество фотографий.
Построение системы для быстрой и точной навигации в миллионах изображений
 --- не тривиальная задача. Одно из самых распространенных решений этой
 проблемы заключается в использовании тегов. Частный случай такого подхода ---
 использование хештегов в социальных сетях.


\indent
\textit{Хештег} --- это любое слово или фраза без пробелов, перед которой стоит 
символ \#, который называется \textit{диез} или \textit{решетка}, а в англоязычном 
варианте -- \textit{hash}, отсюда и название. Приведем несколько примеров:
\htag{masterwork}, \htag{spbu}, \#htag{lovesport}. Обычно в браузере или 
приложении хештег отображается как гипертекст, кликнув по которому можно 
получить список публикаций, снабженных таким же тегом.

\indent
Кроме простоты и удобства теги обладают еще одним полезным свойством 
-- они позволяют не думать об  иерархии структурируемой информации. 
Например, набор изображений можно разложить
по папкам, создав иерархию по датам, геолокациям или авторству. Причем в отдельных
случаях подобрать наиболее подходящую иерархию бывает затруднительно. Проблему
можно решить так: достаточно поставить несколько тегов для всех изображений,
а сами они могут храниться в плоской системе файлов.
Благодаря этому свойству тегирование используются  для рубрикации контента 
не только только онлайн, но и в оффлайн приложениях, например, 
просмоторщиках фотографий. 


\indent
Можно выделить две разновидности популярных хэштегов в социальных сетях.
Первые, используемые недолго и посвященные каким-то социальным явлениям
или событиям, например: \htag{elections2018}, \htag{metoo}. 
И вторые, широкораспространенные, но не связанные с новостной повесткой, 
например: \htag{sport}, \htag{cafe}; они и будут нас интересовать.
 Данная работа посвящена разработке интеллектуальной системы, 
подсказывающей пользователю релевантные хештеги к загружаемым фотографиям.
 Кроме того, с помощью такой системы можно решать и 
  \textquotedblleft обратную\textquotedblright  задачу -- определять,
уместно ли поставлены те или иные теги к заданным изображениям. 
Способность системы давать ответ на такой вопрос можно использовать для 
выявления злоупотреблений со стороны пользователей. Например, зачастую
 в рекламных целях продвигаемые публикации снабжают множеством популярных 
 тегов, не имеющих никакого отношения к публикуемой информации.
 
 