\section{Постановка задачи}


\indent 
\indent
Целью настоящей работы является построение интеллектуальной системы для
структурирования изображений в социальных сетях. А именно, предлагается
автоматизировать процесс добавления пользовательских хештегов к загружаемым
изображениям. С точки зрения машинного обучения решается задача классификации изображений.

\indent
В настоящее время наилучшие результаты в задаче классификации изображений 
удаётся получить, используя глубокие сверточные нейронные сетей. В 
качестве примера можно привести один из самых известных и больших
конкурсов по классификации изображений датасета \textit{ImageNet}\cite{imagenet},
который проводится ежегодно с 2010 года. В настоящее время решения всех призеров 
так или иначе базируются на сверточных нейронных сетях. TODO links

\indent 
Идея применить машинное обучение к тегированию изображений в социальных
сетях не нова, можно привести в пример исследования s1, s2 и s3 TODO links. 
 

 \indent
 В данной работе предлагается способ улучшить точность предсказания
 популярных тегов, сузив информационный домен, к которому
 эти теги относятся.
 Больш\'{а}я часть из наиболее употребимых тегов в социальных сетях описывает
 эмоции, чувства или другие абстрактные понятия, не имеющие прямого выражения
 в объектах реального мира. Например, в 2018 году одними из самых популярных 
 хэштегов в сети \textit{Instagram} стали \htag{love}, \htag{happy},
 \htag{beautiful}. Понятно, что модели компьютерного зрения 
 наоборот будут точнее работать для тегов, связанных с наличием в кадре
 тех или иных сущностей, например
  \htag{beach}, \htag{sky} или \htag{arhcitecture}. Для таких случаев и будет
  строиться модель, описанная в настоящей работе.

\indent  
  В качестве датасета, который может быть использован для наших целей и 
   разметка которого напрямую связана с объектами, 
  находящимися в кадре, был выбран \textit{Scene Understanding Dataset (SUN)}. 
Это набор изображений для каждого из которых выбрана одна
  из четырехсот локаций (сцен), вот несколько примеров:
  
\begin{itemize}
    \item \textit{baseball field}
    \item \textit{basketball court}
    \item \textit{ice shelf}
    \item \textit{forest}
    \item \textit{wind farm}
\end{itemize}

  Большинство названий локаций сами по себе не являются популярными 
  тегами из социальных сетей. Поэтому необходимо
  сопоставить их широко распространенным хэштегам (если это возможно):
  
  \begin{itemize}
      \item \textit{baseball field, basketball court}  $\rightarrow$ \htag{sport}
      \item \textit{ice shelf, forest} $\rightarrow$ \htag{nature}
      \item \textit{wind farm} $\rightarrow$ ?
  \end{itemize}
  
  Выполнив сопоставление, можно присутпить к написанию
  и обучению сверточной нейронной сети. В итоге будет получена
  модель, которая по окружению, обнаруженном на пользовательском фото,
  сможет подсказывать подходящий хэштег.
  
  Ясно, что полученная модель будет корректно работать лишь для
  ограниченного (пусть и большого) домена фотографий. Следовательно, необходимо 
  обучить её распознавать не входящие в этот домен изображения и
   не пытаться определить их категорию. Кроме того, в случае низкой уверенности
   в правильности предсказания так же лучше ничего не делать. 
    По мнению автора, гораздо предпочтительнее не 
   предложить пользователю подходящий хэштег, чем многократно предлагать 
   нерелевантные варианты.
   
Наконец, точность предсказаний обученной модели будет проверена 
вручную группой пользователей на выборке реальных фотографий из 
социальных сетей.
   
Научная новизна работы определяется:
\begin{itemize}
    \item Адаптацией датасета \textit{SUN} для решения задачи о структурировании
    изображений в социальной сети;
    \item Исследованием новых схем обучения неронных сетей, предложенных автором. 
\end{itemize}






